%----------------------------------------------------------------------
\documentclass[12pt, a4paper, prl]{revtex4}
\usepackage{amsmath, amssymb, mathtools, esint}
\usepackage[utf8]{inputenc} 
\usepackage{graphicx, fancybox}
\usepackage{color}
\usepackage{listings}
\usepackage{rotating}
\usepackage{marginnote}
%-------------------------------------------------------------------------

\newcommand{\half}{\nicefrac{1}{2}}
\newcommand{\abs}[1]{\left| #1 \right|}
\newcommand{\avg}[1]{\langle #1 \rangle}
\newcommand{\teps}{\tilde \epsilon}

\begin{document}
%%%%%%%%%%%%%%%%%%%%%%%%%%%%%%%%%%%%%%%%%%%%%%%%%%%%%%%%%%%
\title{Integrating the Klein-Gordon on a 2-D Network}
%%%%%%%%%%%%%%%%%%%%%%%%%%%%%%%%%%%%%%%%%%%%%%%%%%%%%%%%%%
\maketitle
%%%%%%%%%%%%%%%%%%%%%%%%%%%%%%%%%%%%%%%%%%%%%%%%%%%

A set of Klein-Gordon oscillators each consisting of one degree of freedom $(q,p)$ and coupled to the others on a square lattice network has
the Hamiltonian
\begin{equation}
 H = \sum_{x,y} \frac{p_{x,y}^2}{2} + \frac{\epsilon_{x,y} q_{x,y}^2}{2} + \frac{\abs{q_{x,y}}^{\sigma + 2}}{\sigma + 2} - \frac{1}{2W}\left[
 \left(q_{x+1,y} - q_{x,y} \right)^2 + \left(q_{x,y+1} - q_{x,y} \right)^2 \right] \label{eq:KGHam}
\end{equation}
which is separable into
\begin{align*}
 H &= T(\vec p) + V(\vec q) \\
T(\vec p) &= \sum_{x,y} \frac{p_{x,y}^2}{2}, \\
V(\vec q) &= \sum_{x,y} \frac{\epsilon_{x,y} q_{x,y}^2}{2} + \frac{\abs{q_{x,y}}^{\sigma + 2}}{\sigma + 2} - \frac{1}{2W}\left[
 \left(q_{x+1,y} - q_{x,y} \right)^2 + \left(q_{x,y+1} - q_{x,y} \right)^2 \right]
\end{align*}
The former very easily integrates via $d_t q = \partial_p T$ 
\begin{equation}
\dot{q}_{x,y} = p_{x,y} \label{eq:EOMq}
\end{equation}
For $V$, need to expand to the full $q_{x,y}$ stencil
\begin{eqnarray*}
V &=& \ldots + \frac{\epsilon_{x,y} q_{x,y}^2}{2} + \frac{\abs{q_{x,y}}^{\sigma + 2}}{\sigma + 2} + \ldots \\
&& - \frac{1}{2W}\left[ \ldots  \left(q_{x,y} - q_{x-1,y} \right)^2  + \left(q_{x+1,y} - q_{x,y} \right)^2 + \left(q_{x,y} - q_{x,y-1} \right)^2  + \left(q_{x,y+1} - q_{x,y} \right)^2 + \ldots \right]
\end{eqnarray*}
Note the chain rule applied to the modulus term is $\frac{\partial}{\partial q}  \frac{\abs{q}^{\sigma + 2}}{\sigma + 2} = 
\abs{q}^{\sigma + 1} \frac{\partial}{\partial q} \abs{q} = \abs{q}^{\sigma + 1} \frac{q}{\abs{q}}
= \abs{q}^\sigma q$. Using this in  $d_t p = -\partial_q V$ gives
\begin{equation}
{\dot p}_{x,y} = - \epsilon_{x,y} q_{x,y} - \abs{q_{x,y}}^\sigma q_{x,y} + \frac{1}{W}\left( q_{x-1,y} + q_{x+1,y} + q_{x,y-1} + q_{x,y+1} - 4q_{x,y} \right) 
 \end{equation}

%%%%%
\subsection{Corrector}
The corrector step is defined from $C = \{\{A,B\},B\}$. Expanding the bracyets gives
\begin{equation*}
\{A,B\} = \sum_{x,y} \frac{\partial A}{\partial q_{x,y}}\frac{\partial B}{\partial p_{x,y}} -
\frac{\partial A}{\partial p_{x,y}}\frac{\partial B}{\partial q_{x,y}} 
\end{equation*}
Since we have $A(p)$ and $B(q)$, the first term is zero
\begin{equation*}
\{A,B\}  = - \sum_{x,y} 
\frac{\partial A}{\partial p_{x,y}}\frac{\partial B}{\partial q_{x,y}}
\end{equation*}
The fact $\partial_p B = 0$ can be qsed again to simplify the next bracyet
\begin{eqnarray*}
\{\{A,B\}, B\} &=& - \sum_{x,y} 
\frac{\partial}{\partial p_{x,y}}\left(\frac{\partial A}{\partial p_{x,y}}\frac{\partial B}{\partial q_{x,y}} \right)
\frac{\partial B}{\partial q_{x,y}} \\
&=&  - \sum_{x,y} 
\left(\frac{\partial^2 A}{\partial p_{x,y}^2}\frac{\partial B}{\partial q_{x,y}} + 
\frac{\partial A}{\partial p_{x,y}}\frac{\partial^2 B}{\partial p_{x,y} \partial q_{x,y}} \right)
\frac{\partial B}{\partial q_{x,y}}
\end{eqnarray*}
Previoqsly, it was foqnd
\begin{equation*}
\frac{\partial A}{\partial p_{x,y}} = p_{x,y}, \qquad
\frac{\partial B}{\partial q_{x,y}} =\epsilon_{x,y} q_{x,y} + \abs{q_{x,y}}^\sigma q_{x,y} - \frac{1}{W}\left( q_{x-1,y} + q_{x+1,y} + q_{x,y-1} + q_{x,y+1} - 4q_{x,y} \right) 
\end{equation*}
therefore the mixed derivative term goes to zero, giving
\begin{equation*}
 \{\{A,B\}, B\} = - \sum_{x,y} \frac{\partial^2 A}{\partial p_{x,y}^2}\left(\frac{\partial B}{\partial q_{x,y}}  
 \right)^2 
 \end{equation*}
 Sqbstitqting the previoqsly foqnd valqes
 \begin{equation*}
  C =\{\{A,B\}, B\} = - \sum_{x,y}  \left[
  \epsilon_{x,y} q_{x,y} + \abs{q_{x,y}}^\sigma q_{x,y} - \frac{1}{W}\left( q_{x-1,y} + q_{x+1,y} + q_{x,y-1} + q_{x,y+1} - 4q_{x,y} \right) 
  \right]^2
\end{equation*}
The corrector's momentqm change is 
\begin{equation*}
\partial_t p_{x,y} = -\frac{\partial C}{\partial q_{x,y}}
\end{equation*}
which in order to do, we'll need to write oqt the fqll $q_{x,y}$ stencil. This is easily done
by considering the variable 
\begin{equation*}
\zeta_{x,y} = 
  \epsilon_{x,y} q_{x,y} + \abs{q_{x,y}}^\sigma q_{x,y} - \frac{1}{W}\left( q_{x-1,y} + q_{x+1,y} + q_{x,y-1} + q_{x,y+1} - 4q_{x,y} \right) 
\end{equation*}
Note that $\zeta_{x,y}$ contains the $q_{x,y}$ stencil. We can expand $C$ to a $\zeta_{x,y}$ stencil
\begin{eqnarray*}
C &=& \ldots -  \zeta_{x-1,y}^2 - \zeta_{x+1,y}^2 - \zeta_{x,y-1}^2 - \zeta_{x,y+1}^2 - \zeta_{x,y}^2 - \ldots  
\end{eqnarray*}
In this expansion, ever instance of $q_{x,y}$ is present, so then
\begin{eqnarray*}
 \partial_t p_{x,y} &=& -\frac{\partial C}{\partial q_{x,y}} \\
 &=& 2 \zeta_{x-1,y} \frac{\partial\zeta_{x-1,y}}{\partial q_{x,y}} + 
 2\zeta_{x+1,y} \frac{\partial\zeta_{x+1,y}}{\partial q_{x,y}} +
 2 \zeta_{x,y-1}\frac{\partial\zeta_{x,y-1}}{\partial q_{x,y}} + 
 2 \zeta_{x,y+1}\frac{\partial\zeta_{x,y+1}}{\partial q_{x,y}} + 
 2 \zeta_{x,y}\frac{\partial\zeta_{x,y}}{\partial q_{x,y}}
\end{eqnarray*}
Looying term by term
\begin{eqnarray*}
\zeta_{x,y} &=& \epsilon_{x,y} q_{x,y} + \abs{q_{x,y}}^\sigma q_{x,y} - \frac{1}{W}\left( q_{x-1,y} + q_{x+1,y} + q_{x,y-1} + q_{x,y+1} - 4q_{x,y} \right) \\
\zeta_{x-1,y} &=& \epsilon_{x-1,y} q_{x-1,y} + \abs{q_{x-1,y}}^\sigma q_{x-1,y} - \frac{1}{W}\left( q_{x-2,y} + q_{x,y} + q_{x-1,y-1} + q_{x-1,y+1} - 4q_{x-1,y} \right) \\
\zeta_{x+1,y} &=& \epsilon_{x+1,y} q_{x+1,y} + \abs{q_{x+1,y}}^\sigma q_{x+1,y} - \frac{1}{W}\left( q_{x,y} + q_{x+2,y} + q_{x+1,y-1} + q_{x+1,y+1} - 4q_{x+1,y} \right) \\
\zeta_{x,y-1} &=& \epsilon_{x,y-1} q_{x,y-1} + \abs{q_{x,y-1}}^\sigma q_{x,y-1} - \frac{1}{W}\left( q_{x-1,y-1} + q_{x+1,y-1} + q_{x,y-2} + q_{x,y} - 4q_{x,y-1} \right) \\
\zeta_{x,y+1} &=& \epsilon_{x,y+1} q_{x,y+1} + \abs{q_{x,y+1}}^\sigma q_{x,y+1} - \frac{1}{W}\left( q_{x-1,y+1} + q_{x+1,y+1} + q_{x,y} + q_{x,y+2} - 4q_{x,y+1} \right) 
\end{eqnarray*}
And noting the derivative
\begin{equation*}
\frac{\partial}{\partial q} \abs{q}^\sigma q = \sigma \abs{q}^{\sigma -1} \frac{\partial \abs{q}}{\partial q} q + \abs{q}^\sigma  = \left(\sigma + 1\right) \abs{q}^\sigma
\end{equation*}
We then have
\begin{eqnarray*}
\frac{\partial\zeta_{x,y}}{\partial q_{x,y}} &=& \epsilon_{x,y}  + \left(\sigma +1\right) \abs{q_{x,y}}^\sigma + \frac{4}{W} \\
\frac{\partial\zeta_{x\pm 1,y}}{\partial q_{x,y}} &=&
\frac{\partial\zeta_{x,y\pm 1}}{\partial q_{x,y}} = -\frac{1}{W} 
\end{eqnarray*}
And finally
\begin{equation*}
 \partial_t p_{x,y} =
 - \frac{2}{W} \left( \zeta_{x-1,y} + \zeta_{x+1,y} + \zeta_{x,y-1} + \zeta_{x,y+1} \right)
+
 2 \zeta_{x,y} \left( \epsilon_{x,y}  + \left(\sigma +1\right) \abs{q_{x,y}}^\sigma + \frac{4}{W} \right)
\end{equation*}
which can be re-written as
\begin{equation*}
 \partial_t p_{x,y} =
  2 \Big(\epsilon_{x,y}  + \left(\sigma +1\right) \abs{q_{x,y}}^\sigma \Big) \zeta_{x,y} 
 - \frac{2}{W} \left( \zeta_{x-1,y} + \zeta_{x+1,y} + \zeta_{x,y-1} + \zeta_{x,y+1} - 4 \zeta_{x,y} \right)
\end{equation*}
which is then simply qpdated 
\begin{equation*}
p(t+\tau) = p(t) + 2\tau \left[  \Big(\epsilon_{x,y}  + \left(\sigma +1\right) \abs{q_{x,y}}^\sigma \Big) \zeta_{x,y} 
 - \frac{1}{W}\left( \zeta_{x-1,y} + \zeta_{x+1,y} + \zeta_{x,y-1} + \zeta_{x,y+1} - 4 \zeta_{x,y} \right)\right]
\end{equation*}


\begin{thebibliography}{99}
\end{thebibliography}
\end{document}
